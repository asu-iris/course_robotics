\documentclass[10pt]{article}
\usepackage[margin=0.8in]{geometry}
\usepackage[utf8]{inputenc}
\usepackage[T1]{fontenc}
\usepackage{graphicx}
\usepackage[export]{adjustbox}
\graphicspath{ {./lecture2-images/} }
\usepackage{amsmath}
\usepackage{amsfonts}
\usepackage{amssymb}
\usepackage[version=4]{mhchem}
\usepackage{stmaryrd}
\usepackage{bbold}
\usepackage{fixltx2e}
\usepackage{caption}
\usepackage{mathtools}
\usepackage[parfill]{parskip}
\usepackage{float}

\begin{document}



\title{Lecture 15: Inverse Velocity Kinematics}
\date{Oct. 12, 2023}
\author{Wanxin Jin}
\maketitle




Suppose that a robot current configuration is $\boldsymbol{q}\in\mathbb{R}^n$, and the motion of the end-effector is assigned to $\boldsymbol{v}_{e}\in\mathbb{R}^r$. We want to find the corresponding joint velocity $\dot{\boldsymbol{q}}$ using the Jacobian

$$
 \boldsymbol{v}_{e}=\boldsymbol{J}(\boldsymbol{q})\dot{\boldsymbol{q}}
$$

This problem is corned as \emph{inverse velocity kinematics}.



\section{Non-redundant Manipulator}

When the  manipulator is non-redundant $r=n$ and $\boldsymbol{q}$ is not a singular configuration, one can directly inverse the Jacobian 

$$
 \dot{\boldsymbol{q}}=\text{inv}(\boldsymbol{J}(\boldsymbol{q})) \boldsymbol{v}_{e}
 $$

\section{Redundant Manipulator}

When the manipulator is redundant $(r<n)$, the Jacobian matrix has more columns than rows, and thus there are infinite solutions $\dot{\boldsymbol{q}}$. We need to resort to optimization to pick the best solution  $\dot{\boldsymbol{q}}$ in some sense! 

\subsection{Minimal Norm Selction}

Once the end-effector velocity $\boldsymbol{v}_{e}$ and Jacobian $\boldsymbol{J}$ are given, we establish the following optimization to pick $\dot{\boldsymbol{q}}$

$$
    \begin{aligned}
        \min_{\dot{\boldsymbol{q}}} \quad &\frac{1}{2} \dot{\boldsymbol{q}}^{T} \boldsymbol{W} \dot{\boldsymbol{q}}\\
        \text{s.t.}\quad & \boldsymbol{v}_{e}  =\boldsymbol{J}(\boldsymbol{q}) \dot{\boldsymbol{q}}
    \end{aligned}
    $$

This means that we want to pick the best $\dot{\boldsymbol{q}}$ in the sense of having the minimal norm of joint velocities.
By some derivation, if  $\boldsymbol{q}$ is not a singular configuration, the optimal solution to the above optimization is

$$
\dot{\boldsymbol{q}}=\boldsymbol{W}^{-1} \boldsymbol{J}^{T}\left(\boldsymbol{J} \boldsymbol{W}^{-1} \boldsymbol{J}^{T}\right)^{-1} \boldsymbol{v}_{e} .
$$



In particular, if $\boldsymbol{W}=\boldsymbol{I}$, the solution simplifies into

$$
\dot{\boldsymbol{q}}=\boldsymbol{J}^{\dagger} \boldsymbol{v}_{e} \quad \text{with} \quad \boldsymbol{J}^{\dagger}=\boldsymbol{J}^{T}\left(\boldsymbol{J} \boldsymbol{J}^{T}\right)^{-1}
$$


is the right pseudo-inverse of $\boldsymbol{J}$.

\subsection{Close-to-Reference Selction}

We have other criteria to pick the best $\dot{\boldsymbol{q}}$. When we are given a reference $\dot{\boldsymbol{q}}_{0}$, we want to pick the best  $\dot{\boldsymbol{q}}$ in the sense of having the minimal distance to  $\dot{\boldsymbol{q}}_{0}$. Thus, we establish the following optimization

$$
    \begin{aligned}
        \min_{\dot{\boldsymbol{q}}} \quad &
        \frac{1}{2}\left(\dot{\boldsymbol{q}}-\dot{\boldsymbol{q}}_{0}\right)^{T}\left(\dot{\boldsymbol{q}}-\dot{\boldsymbol{q}}_{0}\right)\\
        \text{s.t.}\quad & \boldsymbol{v}_{e}  =\boldsymbol{J}(\boldsymbol{q}) \dot{\boldsymbol{q}}
    \end{aligned}
    $$

After doing some derivation,  if $\boldsymbol{q}$ is not a singular configuration, the solution to the above optimization is

$$
\dot{\boldsymbol{q}}=\boldsymbol{J}^{\dagger} \boldsymbol{v}_{e}+\left(\boldsymbol{I}_{n}-\boldsymbol{J}^{\dagger} \boldsymbol{J}\right) \dot{\boldsymbol{q}}_{0} .
$$

The obtained solution has two terms. The first term is the same as the solution of minimizing the joint velocity norm. In the second term,   $\left(\boldsymbol{I}-\boldsymbol{J}^{\dagger} \boldsymbol{J}\right)$ is null space projection matrix $\boldsymbol{P}$ (recall we have mentioned in our previous lecture), which allows the projection of any  $\dot{\boldsymbol{q}}_{0}$ in the null space of $\boldsymbol{J}$.


The final question is how to specify the reference $\dot{\boldsymbol{q}}_{0}$ for a redundant manipulator. A typical choice is

$$
\dot{\boldsymbol{q}}_{0}=k_{0}\left(\frac{\partial w(\boldsymbol{q})}{\partial \boldsymbol{q}}\right)^{T}
$$

where $k_{0}>0$ and $w(\boldsymbol{q})$ is a secondary objective function of the joint variables. Since the solution moves along the direction of the gradient of the objective function, it attempts to maximize it locally the secondary objective. Such a secondary objective can be the following:

(1) The manipulability measure:
  $$
w(\boldsymbol{q})=\sqrt{\operatorname{det}\left(\boldsymbol{J}(\boldsymbol{q}) \boldsymbol{J}^{T}(\boldsymbol{q})\right)}
$$

By maximizing this measure, redundancy is exploited to move away from singularities.

(2) The distance from mechanical joint limits:

  $$
w(\boldsymbol{q})=-\frac{1}{2 n} \sum_{i=1}^{n}\left(\frac{q_{i}-\bar{q}_{i}}{q_{i M}-q_{i m}}\right)^{2}
$$

where $q_{i M}\left(q_{i m}\right)$ is the maximum (minimum) joint limit and $\bar{q}_{i}$ the middle point of the joint range; thus, by maximizing this distance, redundancy is exploited to keep the joint variables close to the center of their ranges.


(3) The distance from an obstacle:

$$
w(\boldsymbol{q})=\min _{\boldsymbol{p}, \boldsymbol{o}}\|\boldsymbol{p}(\boldsymbol{q})-\boldsymbol{o}\|
$$

where $\boldsymbol{o}$ is the position vector of a suitable point on the obstacle. 


\section{Singularity}

The inverse velocity kinematics for both the redundant and non-redundant manipulators require the current robot configuration $\boldsymbol{q}$ is nonsingular, i.e., the Jacobian $\boldsymbol{J}(\boldsymbol{q})$ has full rank.
At a singular configuration,  $\boldsymbol{v}_{e}=\boldsymbol{J} \dot{\boldsymbol{q}}$ contains linearly dependent equations, and it is possible to find a solution $\dot{\boldsymbol{q}}$ only if $\boldsymbol{v}_{e} \in \mathcal{R}(\boldsymbol{J})$. This situation means that the assigned $\boldsymbol{v}_{e}$ is physically achievable by finding a joint velocity $ \dot{\boldsymbol{q}}$, even though it is at a singular configuration. If instead $\boldsymbol{v}_{e} \notin \mathcal{R}(\boldsymbol{J})$, the system of equations has no solution; this means that $\boldsymbol{v}_{e}$ 
 cannot be achievable by the manipulator at the given posture.


Inversion of the Jacobian can represent a serious inconvenience not only at a singularity but also in the neighborhood of a singularity. In the neighborhood of a singularity,  a relatively small $\boldsymbol{v}_{e}$ (in terms of its norm)  which can cause large joint velocities. Consider the example of the shoulder singularity for the anthropomorphic arm. If a path is assigned to the end-effector which passes nearby the base rotation axis, the base joint is forced to make a rotation of about $\pi$ in a relatively short time to allow the end-effector to keep tracking the imposed trajectory. An solution overcoming the problem of inverting differentia kinematics in the neighbourhood of a singularity is provided by the so-called damped least-squares (DLS) inverse

$$
\boldsymbol{J}^{\star}=\boldsymbol{J}^{T}\left(\boldsymbol{J} \boldsymbol{J}^{T}+k^{2} \boldsymbol{I}\right)^{-1}
$$

where $k$ is a damping factor that renders the inversion better conditioned from a numerical viewpoint. It can be shown that such a solution can be obtained by minimizing the following objective

$$
 \min_{\dot{\boldsymbol{q}}}\quad \frac{1}{2}\left(\boldsymbol{v}_{e}-\boldsymbol{J} \dot{\boldsymbol{q}}\right)^{T}\left(\boldsymbol{v}_{e}-\boldsymbol{J} \dot{\boldsymbol{q}}\right)+\frac{1}{2} k^{2} \dot{\boldsymbol{q}}^{T} \dot{\boldsymbol{q}}
$$



\end{document}